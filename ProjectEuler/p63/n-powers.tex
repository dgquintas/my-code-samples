%% LyX 1.6.1 created this file.  For more info, see http://www.lyx.org/.
%% Do not edit unless you really know what you are doing.
\documentclass[english]{article}
\usepackage[T1]{fontenc}
\usepackage[latin1]{inputenc}

\usepackage{babel}
\usepackage{amsmath}
\usepackage{amssymb}

\begin{document}

\section{Problem Text}

The $5$-digit number, $16807=7^{5}$, is also a fifth power. Similarly,
the $9$-digit number, $134217728=8^{9}$, is a ninth power.

How many $n$-digit positive integers exist which are also an $n$th
power? 


\section{Solution}

A number of the form $x^{y}$ has $\lfloor\log x^{y} \rfloor+1$
digits in base $10$. We are asked for how many $x,y \in \mathbb{Z}^{+}$ 
does the following equation hold:

\begin{equation}\label{eq1}
 \lfloor\log x^{y} \rfloor+1=y
\end{equation}

Firstly, extract the exponent from the logarith:

\begin{equation} \label{eq2}
    \lfloor \log x^{y} \rfloor +1 \equiv \lfloor y \cdot \log x \rfloor +1 
\end{equation}


Secondly, get rid of the floor function. By this function's definition:
\begin{equation} \label{eq3}
    \lfloor y \cdot \log x \rfloor \leq  y \cdot \log x  < \lfloor y \cdot \log x \rfloor + 1 
\end{equation}
Note that by \eqref{eq1}, $\lfloor y \cdot \log x \rfloor = y -1$; likewise $\lfloor y \cdot \log x \rfloor + 1 = y$. 
Combined with equation \eqref{eq3}, we have:
\begin{equation} \label{eq4}
    y-1 \leq y \cdot \log x < y
\end{equation}
Dividing the whole expresion by $y$:
\begin{equation} \label{eqFinal}
    1-\frac{1}{y} \leq \log x < 1
\end{equation}

Finally, we have an upper bound for $x$, namely $\log x < 1 \iff x < 10$. 

Regarding the exponent $y$, for all integers $ 0 < x < 10$, \eqref{eqFinal} is
satisfied if and only if $1-\frac{1}{y} \leq \log x$. Solving for $y$, $y < \frac{1}{1-\log x}$.
Given that the exponent must be a positive integer, $\lfloor \frac{1}{1-\log x}\rfloor$ represents
the number of exponents resulting in an $n$-digit number for $0 < x < 10$. 

\paragraph{Python one-liner}
The following Python code calculates the result based on the previous observations.

\begin{verbatim}
from math import log10, floor; 
sum( (floor(1 / (1-log10(i))) for i in xrange(1,10) ) )
\end{verbatim}
The result is $49$. 



\end{document}
